\chapter {Giới thiệu}

\section{Giới thiệu đề tài}
Sự tương tác giữa con người và máy tính ngày càng trở nên quan trọng khi các thiết bị thông minh như điện thoại, máy tính, máy tính bảng ngày càng gia tăng. Khoa học nghiên cứu về nó cũng được chú trọng phát triển theo, trong đó việc phát hiện hướng nhìn của người dùng thông qua đôi mắt trở thành một lĩnh vực trong thị giác máy tính ngày nay.

Qua các hình ảnh thu thập bằng máy quay trên thiết bị thông minh như điện thoại hay máy tính, ta có thể phát hiện được mục tiêu hay đối tượng mà người dùng đang hướng tới màn hình. Ứng dụng này có thể giúp người khuyết tật điều khiển các thiết bị chỉ bằng ánh mắt mà không cần đến đôi tay.

Tương tự như vậy khi gắn một máy quay nhỏ trên xe hơi theo dõi người lái xe, dựa vào hướng nhìn đôi mắt của tài xế mà ta có thể phát hiện kịp thời người này ngủ gật hay không chú ý khi lái xe (khi  ngủ gật mắt sẽ trùng xuống, khi không tập trung mắt sẽ hướng đi nơi khác).

Trong một lớp học hay một buổi dự thảo ta có thể đo được độ tập trung của người tham gia để kịp thời thay đổi nội dung cho phù hợp hơn, tránh lãng phí thời gian của mọi người.

Ngoài ra, việc dự đoán chính xác hướng nhìn của đôi mắt có thể phân tích hành vi mua hàng của khách như khách hay tập trung vào tầm nhìn nào nhiều nhất mà ta có thể sắp xếp mặt hàng quan trọng vừa tầm. Hay đơn giản như việc lướt web của người dùng Internet để sắp xếp nội dung quan trọng hay không quan trọng vào các vị trí phù hợp.

Đó là tầm quan trọng của việc "dự đoán hướng nhìn" ứng dụng vào thực tế mà nhóm tiến hành nghiên cứu và hiện thực đề tài.
\section{Mục tiêu của đề tài}
Mục tiêu của đề tài là nghiên cứu, hiểu và hiện thực một số phương pháp học sâu để phát hiện hướng nhìn của con người qua hình ảnh.

Một số vấn đề đặt ra:
\begin{itemize}
    \item Làm thế nào để giải quyết bài toán trên?
    \item Cách tiếp cận như thế nào?
    \item Những công nghệ nào đã và hiện đang được sử dụng?
    \item Hướng cải tiến?...
\end{itemize}

Như vậy để thực hiện theo đúng mục tiêu của đề tài cần xác định một số công việc phải giải quyết như sau:
\begin{itemize}
    \item Tìm kiếm và thu thập dữ liệu phù hợp với nội dung đề tài.
    \item Tìm hiểu các phương pháp tiếp cận đã được hiện thực
    \item Lựa chọn mô hình phù hợp
    \item Lên kế hoạch hiện thực, phát triển hệ thống nhận diện huấn luyện và kiểm thử.
\end{itemize}
\section{Cấu trúc luận văn}
Trong giai đoạn luận văn đề tài nhóm đã thực hiện được một số công việc liên quan sẽ trình bày trong báo cáo như sau:

\begin{itemize}
    \item Chương 1: Giới thiệu tổng quan về nhận diện hướng nhìn, cũng là chương hiện hành. Trong chương này sẽ đưa đến cái nhìn tổng quát về đề tài, tiềm năng và ứng dụng thực tế trong tương lai.
    \item Chương 2: Tổng quan một số công trình nghiên cứu liên quan tới đề tài mà nhóm tìm hiểu được qua ba phần: hướng tiếp cận, mô hình sử dụng và kết quả đạt được.
    \item Chương 3: Các kiến thức nền tảng phục vụ như mạng nơ-ron tích chập trong học sâu (Deep Learning), các kiến trúc mạng nâng cao của mạng tích chập.
    \item Chương 4: Trình bày hướng tiếp cận bài toán: chương này sẽ đi vào chi tiết cách tiếp cận thực hiện mô hình, bao gồm công cụ sử dụng.
    \item Chương 5: Tổng kết những công việc nhóm đã làm được, đánh giá và định hướng kế hoạch mà nhóm tiếp tục phát triển trong luận văn.
\end{itemize}
